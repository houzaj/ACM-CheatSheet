\noindent\textbf{Prerequisite}

$f$ is multiplicative function, and $f(p)$ is a low-degree polynomial, which can be calculated in quick time when $p$ is a prime.  

\noindent\textbf{Formula}

\begin{equation*}
\begin{aligned}
prime   &= \left\{p_1, p_2, \cdots, p_j\right\} \\
sum_{j} &= \sum_{i=1}^{j} f(p_i) \\
g(n, j) &= \sum_{i=2}^{n} f(i) \cdot [i \in prime \ or \ min(p) > p_j, p|i, p \in prime] \\
&= g(n, j - 1) - f(p_j) \cdot \left(g\left(\left\lfloor\frac{n}{p_j}\right\rfloor, j-1\right) - sum_{j-1}\right) \\
S(n, j) &= \sum_{i=2}^{n} f(i) \cdot [min(p) > p_j, p|i, p \in prime] \\
&= g(n, \infty) - sum_j + \sum_{e} \sum_{k=j+1} f(p_k^e) \cdot \left( S\left(\left\lfloor \frac{n}{p_k^e}\right\rfloor, k \right) + \left[e \neq 1\right] \right) \\
ans &= \sum_{i=1}^{n}f(i) = S(n, 0) + f(1)\\
\end{aligned}
\end{equation*}

